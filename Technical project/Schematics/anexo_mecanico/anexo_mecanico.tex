\documentclass[a4paper ,12pt, onecolumn]{article}
\usepackage[utf8]{inputenc}
\usepackage[spanish]{babel}
\usepackage[hidelinks]{hyperref}
\usepackage{graphicx}
\begin{document}
\title{Anexo diseño mecánico}

\author{Rubén Arce}
\date{\today}
\maketitle
\cleardoublepage
\tableofcontents
\cleardoublepage

\section{Introducción}
Para llevar a cabo el diseño mecánico de estas piezas concretas se ha optado por emplear Freecad, un
programa de software libre que permite hacer modelos sencillos de forma gratuita.
Se ha empleado la versión 12 del mismo corriendo en una raspberry pi 4 de 4Gb de RAM, este es el único
programa que corre en linux de diseño 3D y además consume pocos recursos.
\paragraph{}
Los primeros prototipos se han llevado a cabo con una impresora 3D Anet A8 con el firmware de Marling
2.0 y como material PLA de 1,75mm y distancia entre capa de 0,2 mm.
\section{Emisor beacon}
    \subsection{Aspectos a considerar en el diseño}
        \begin{enumerate}
            \item Estéticamente atractivo
            \item Pequeñas dimensiones
            \item Cómodo para llevar colgado del cuello
            \item Facilidad para desmontar y recargar las baterías o pilas
        \end{enumerate}
    \subsection{Planos y dimensiones}
        Una vez llevado a cabo el siseño se han obtenido los planos y se ha llevado a cabo la impresión 3d de los mismos.
        \includegraphics[scale=0.25]{../receiver_1.PNG}
    \subsection{Imágenes renderizado}
    \begin{center}
        \includegraphics[scale=0.25]{../receiver_1.PNG}
        \includegraphics[scale=0.25]{../receiver_2.PNG}
    \end{center}
    \subsection{Imágenes reales}
        \includegraphics[scale=0.2]{../receiver_1.PNG}
        \includegraphics[scale=0.2]{../receiver_2.PNG}

\section{Receptor beacon o gateway}
    \subsection{Aspectos a considerar en el diseño}
        \begin{enumerate}
            \item Estéticamente atractivo
            \item Pequeñas dimensiones
        \end{enumerate}
    \subsection{Planos y dimensiones}
    \subsection{Imágenes renderizado}   
        \includegraphics[scale=0.2]{../receiver_1.PNG}
        \includegraphics[scale=0.4]{../receiver_2.PNG}
    \subsection{Imágenes reales}
        \includegraphics[scale=0.2]{../receiver_1.PNG}
        \includegraphics[scale=0.4]{../receiver_2.PNG}

\section{Conclusiones}
    Tras desarrollar los primeros prototipos se procedió a llevar a cabo una prueba real con los mismos 
    en un entorno industrial para comprobar si era necesaria una versión 2.0 de los mismos, los resultados
    dejaron claro que este primer diseño era el acertado.
\end{document}