\documentclass[a4paper ,12pt, onecolumn]{article}
\usepackage[utf8]{inputenc}
\usepackage[spanish]{babel}
\usepackage[hidelinks]{hyperref}
\usepackage{graphicx}
\begin{document}
\title{Anexo diseño mecánico}

\author{Rubén Arce}
\date{\today}
\maketitle
\cleardoublepage
\tableofcontents
\listoffigures
\cleardoublepage

\section{Introducción}
Para llevar a cabo el diseño mecánico de estas piezas concretas se ha optado por emplear Freecad, un
programa de software libre que permite hacer modelos sencillos de forma gratuita.
Se ha empleado la versión 12 del mismo corriendo en una raspberry pi 4 de 4Gb de RAM, este es el único
programa de diseño 3D que corre en Linux  y además consume pocos recursos.
\paragraph{}
Los primeros prototipos se han llevado a cabo con una impresora 3D Anet A8 con el firmware de Marling
2.0 y como material PLA de 1,75mm, la velocidad de impresión ha sido de 40 mm/s.
\paragraph{}
El programa slicer empleado ha sido Cura en su versión v4.4, para los prototipos se ha empleado un relleno del 5\% y distancia
entre capas de 0,3mm. Para la versión del cliente se ha optado por un relleno del 40\% y una distancia entre capas de 0,12mm.
\section{Emisor beacon}
    \subsection{Aspectos a considerar en el diseño}
        Antes de afrontar el apartado de diseño se ha optado por reunir las condiciones indispensables para 
        conseguir un modelo acorde a las necesidades del cliente.
        \begin{enumerate}
            \item Estéticamente atractivo
            \item Pequeñas dimensiones
            \item Cómodo para llevar colgado del cuello
            \item Facilidad para desmontar y recargar las baterías o pilas
            \item Limitaciones en el tamaño de máximo 220x220x250 mm debido al volumen de impresión.
        \end{enumerate}
    \subsection{Planos y dimensiones}
        Una vez llevado a cabo el diseño se han obtenido los planos y se ha llevado a hecho la impresión 3d de los mismos.
        \begin{center}
            \begin{figure}[h]
                \centering
                \includegraphics[width=0.6\textwidth]{../model_beacon.PNG}
                \caption{Plano y dimensiones del Beacon}
                \label{fig:mesh1}
            \end{figure}
        \end{center}
    \subsection{Imágenes renderizado}
        Obtenemos el ensablaje desde el software cad:
        \begin{center}
            \begin{figure}[h]
                \centering
                \includegraphics[width=0.3\textwidth]{../mechanical_beacon.PNG}
                \caption{Renderizado de la PCB}
                \label{fig:mesh1}
            \end{figure}
        \end{center}

    \subsection{Imágenes reales}
        Tras verificar el diseño en el ordenador se carga el filamento y se procede a llevar a cabo la impresión, un par de horas más tarde
        obtengo lo siguiente:
        \paragraph{}
        \begin{center}
            \begin{figure}[h]
                \centering
                \includegraphics[width=0.36\textwidth]{../3d_beacon_1.jpeg}
                \includegraphics[width=0.4\textwidth]{../3d_beacon_2.jpeg}
                \caption{Imagen real del Beacon con batería}
                \label{fig:mesh1}
            \end{figure}
        \end{center}

\section{Receptor beacon o gateway}
    \subsection{Aspectos a considerar en el diseño}
        De nuevo se apuesta por reunir las especificaciones que ha de tener el contenedor de la electrónica:
        \begin{enumerate}
            \item Estéticamente atractivo, puesto que va a estar físicamente a la vista.
            \item Pequeñas dimensiones y discreto.
            \item Limitaciones en el tamaño de máximo 220x220x250 mm debido al volumen de impresión.
            \item Estructura sólida que garantize un buen agarre de la PCB en vertical.
        \end{enumerate}
        \paragraph{}
        Se ha de tener en cuenta que este equipo se encontrará en un lugar elevado atornillado o pegado a la pared, 
        es por ello por lo que he de hacer una caja robusta en la que no entre suciedad. 
    \subsection{Planos y dimensiones}
        \paragraph{}
        Partiendo de las dimensiones de la tarjeta que se muestra a continuación:
        \begin{center}
            \begin{figure}[h]
                \centering
                \includegraphics[width=0.55\textwidth]{../model_master.PNG}
                \caption{Plano caja del emisor}
                \label{fig:mesh1}
            \end{figure}
        \end{center}
        \paragraph{}
        \paragraph{}

    \subsection{Imágenes renderizado}
        Los renderizados 3D previos a la impresión:
        \begin{center}
            \begin{figure}[h]
                \centering
                \includegraphics[width=0.55\textwidth]{../mechanical_master.PNG}
                \caption{Renderizado caja del emisor}
                \label{fig:mesh1}
            \end{figure}
        \end{center}
        \paragraph{}

    \subsection{Imágenes reales}
        Una vez impresas las piezas y sin necesidad de lijar o ajustar algún apriete o tolerancia:
        \begin{center}
            \begin{figure}[h]
                \centering
                \includegraphics[width=0.3\textwidth]{../3d_master_1.jpeg}
                \includegraphics[width=0.3\textwidth]{../3d_master_2.jpeg}
                \caption{Caja real del receptor con dimensiones}
                \label{fig:mesh1}
            \end{figure}
        \end{center}
        \paragraph{}
        Introduciendo la electrónica dentro podemos ver el resultado completo:
        \begin{center}
            \begin{figure}[h]
                \centering
                \includegraphics[width=0.58\textwidth]{../3d_master_2_box.jpeg}
                \includegraphics[width=0.4\textwidth]{../3d_antenna.jpeg}
                \caption{Caja real del receptor con electrónica dentro}
                \label{fig:mesh1}
            \end{figure}    
        \end{center}
\section{Conclusiones}
    Tras desarrollar los primeros prototipos se procedió a llevar a cabo una prueba real con los mismos 
    en un entorno industrial para comprobar si era necesaria una versión 2.0 de los mismos, los resultados
    dejaron claro que este primer diseño era el acertado.

    Execeptuando una caida accidental todos los diseños sobrevivieron a la prueba, las imágenes dentro de la planta
    no han sido posibles de llevar a cabo por seguridad y ausencia de permisos por parte de administración de la empresa.

    Igualmente lo equipos de las fotos anteriores fueron los empleados y se aseguraron a unas estanterías con cinta de doble cara.

\end{document}