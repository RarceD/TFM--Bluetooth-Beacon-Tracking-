\documentclass[a4paper ,12pt, onecolumn]{article}
\usepackage[utf8]{inputenc}
\usepackage[spanish]{babel}

\usepackage[hidelinks]{hyperref}
\usepackage{graphicx}
\graphicspath{ {./images/} }


\begin{document}

\title{Sistemas de posicionamiento de objetos mediante tecnología Bluetooth Low Energy
Beacon }
\author{Rubén Arce}
\date{\today}
\maketitle
\cleardoublepage
\tableofcontents
\cleardoublepage
\section{Memoria descriptiva}
Antecedentes, objeto, normativa, reglamentación
Extensión máxima de 30 páginas + anexos 60
Defensa de 20 min  40
    \subsection{Antecedentes y objeto del proyecto}
        Este proyecto surge como solución a una necesidad de poder localizar un número elevado de equipos en 
        constante movimiento con cierta exactitud en un espacio cerrado.
        \begin{enumerate}
            \item Estudio sobre las distintas alternativas para llevar a cabo este tracking de objetos de una forma
            sencilla y sin requerir de una inversión grande.
            \item Análisis de las tecnologías existentes para la monitorización en interiores.
            \item  Estudio del parámetro RSSI como medida de la potencia de una señal y cálculo de la distancia entre equipos.
            \item Desarrollo de un sistema de visualización mediante un mapa sobre el que poder ver los equipos en movimiento.
            \item Diseño de hardware específico para la aplicación requerida.
            \item Programación tanto del equipo emisor como del receptor así como del algoritmo de visualización.
            \item Pruebas de campo y comprobación del rendimiento del equipo.
        \end{enumerate}
    \subsection{Ambito de aplicación}
        El tracking de objetos en espacios cerrados está día a día incrementando su popularidad debido a que es una 
        arma de propaganda muy poderosa que demandan los grandes hipermercados para llevar a cabo estudios de marketing
        y poder analizar el comportamiento de los clietes.
        \paragraph{}
        Los principales ámbitos de aplicación de esta tecnología son:
        \begin{enumerate}
            \item Marketing: Estudios de mercado y de necesidad de los clientes. "heat map"
            \item Seguridad: Tan solo los empleados con autorización y proximidad podrán llevar a cabo acciones.
            \item Vandalismo: Conociendo la localización de los equipos dentro de un local cerrado en el momento 
            en el que se deja de situar un elemento se puede dar la voz de alarma ante un robo.
            \item Anuncios y nueva forma de publicidad: Activar acciones en función de la localización en determinados puntos de interés
        \end{enumerate}
    \subsection{Análisis de soluciones}
        \subsubsection {Wifi}
        La posibilidad de llevar a cabo el tracking mediante la escucha de wifis es una opción, pero para ello sería
        necesario en primer lugar que se colocara un móvil en cada uno de los objetos a identificar. Esto es aceptable 
        en el caso de monitorización de personas pero inasumible para objetos.
        \paragraph{}
        El procedimiento consistiría en llevar a cabo un barrido de direcciones MACs que generan los móviles cuando tienen el wifi dado,
        es decir que esto no funcionaría en el caso de que una persona lleve puestos los datos únicamente.
        \paragraph{}
        Otra desventaja es el hecho de que al tener un equipo buscando wifis la información personal del propietario como localización del móvil,
        la fecha y la hora a la que se intentó conectar y la dirección MAC que identifica y relaciona un móvil a una persona. Se han de tener 
        en cuenta la política de privacidad empleando esta tecnología para llevar a cabo el tracking.
        \paragraph{}
        \includegraphics[scale=0.4]{WifiCounting.jpg}
        \paragraph{}
        La única solución al problema es que se ha de pedir permiso formal a la persona de alguna forma para poder llevar a cabo
        el tratamiento de sus datos, cosa que es dificil puesto que se desconoce quien va a llevar a cabo el uso del equipo.
        \paragraph{}
        El artículo 29 de privacidad de datos de la comunidad europea dice lo siguiente:
        “WiFi-tracking, depending on the circumstances and purposes of the data collection, such tracking under the GDPR is likely
        either to be subject to consent, or may only be performed if the personal data collected is anonymised.”
        \subsubsection {GPS}
        La tecnología por excelencia para llevar a cabo el seguimiento de objetos o personas es el GPS, la principal desventaja
        que presenta es el elevado consumo energético comparado con el wifi o el bluetooth.
        \paragraph{}
        Consumo energético: Mantener un módulo GPS encendido y pretender llegar a una autonomía de años es a día de hoy
        imposible, es por ello por lo que o se emplea un teléfono móvil para llevarlo a cabo, cosa inviable si se pretende 
        monitorizar objetos en movimiento y no personas, o el tiempo de refresco de los datos ha de ser muy lento, del orden de horas,
        cosa de nuevo inviable para la aplicación que se tiene entre manos.
        \paragraph{}
        Se ha de tener en cuenta también el aumento de precio que supondría un módulo GPS sumado a la antena que lleva consigo de 
        dimensiones nada despreciables.
        \subsubsection {Bluetooth Low Energy - Beacon}
        Existen varios tipos de beacons:
        https://accent-systems.com/es/producto/ibks-105/
        \begin{enumerate}
            \item iBeacon: Fue la primera tecnología BLE Beacon desarrollada por Apple, permite leer y emitir en modo 
            broadcast para cualquier dispositivo que disponga de Bluetooth low energy. Es un protocolo propietario, es 
            decir es un estandar cerrado. 
            Los iBeacons disponen de los siguientes identificadores:
            \begin{itemize}
                \item UUID: Se basan en enviar el indentificador único de dispositivo, una cadena de
                carácteres de 16 bytes que permite caracterizar a caba equipo.
                \item Major: Número entero de 0 a 65535, se usa para identificar grupos, un ejemplo sería 
                asignar un Major común para todos los beacons de una misma planta o habitación.
                \item Minor: Es también un número entero de 0 a 65535 que se emplea para distinguir un beacon
                específico dentro de un grupo, entendiéndose como grupo aquellos beacons con mismo valor de Major.
            \end{itemize}
            \paragraph{}
            \includegraphics[scale=0.5]{tipos_beacon_ibeacon.PNG}
            \item Eddystone: Creado por Google es un protocolo de código abierto. A diferencia del protocolo anterior este 
            permite transmitir:
            \begin{itemize}
                \item URL: Un url propio de una web, de esta forma se evita la necedad de tener que contar con una app instalada.
                \item UID: Similar al UUID del iBeacon, este parámetro identifica al beacon y permite llevar acciones individuales.
                \item TML: Permite enviar información relativa al beacon como por ejemplo el porcentaje de batería o valores de sensores.
            \end{itemize}
            \paragraph{}
            \includegraphics[scale=0.5]{tipos_beacon_edison.PNG}
            \item AltBeacon: Es un protocolo de código abierto que surge como resultado de las incompatibilidades de las dos 
            tecnologías anteriores, la ventaja principal es que permite más flexibilidad de modificaciones así como compatibilidad 
            entre sistemas operativos.
            \paragraph{}
            \includegraphics[scale=0.5]{tipos_beacon_altbeacon.PNG}
        \end{enumerate}
    \subsection{Resultados finales}
    \subsection{Planificación}
        \paragraph{}
        Para sintetizar el proceso de desarrollo del proyecto técnico se empleará un diagrama GANTT: 
        \paragraph{}
        \includegraphics[width=15cm, height=8cm]{gantt.PNG}
        Fase 1: Analisis de la problemática y búsqueda de soluciones
        Fase 2: Prototipos y programación
        Fase 3: Pruebas de campo
        Fase 4: Evolución del equipo en instalación y depuración 
\section{Memoria justificativa}
    \subsection{Cálculos justificativos de la instalación}
        Calculos de consumos de los equipos y dimensionamiento de la batería.
        Los datos de partida son el consumo medio y duración del batería durante 2 años.
        \subsubsection{Cálculo de distancia por RSSI}
            \paragraph{}
            El RSSI (Received signal strength indicator) es un indicador de la energía o potencia recibida en un mensaje de radio, 
            está asociado con la atenuación de la señal, cuanto más pequeña es su valor menor atenuación.Este valor está 
            presente no solo en Bluetooth sino también en el Wifi (2,4 GHz)o en las bandas de radio industriales, científicas y médicas,
            las denominadas ISM (desde 433MHz a 458.5MHz y desde 860MHz a 960MHz)
            \paragraph{}
            De las muestras obtenidas se puede concluir que se puede llegar a estimar la distancia a partir de los valores
            de rrsi con un error que disminuye cuanto más alejados son los elementos a medir. 
            \paragraph{}
            Los rangos del RRSI se obtiene bien por aproximaciones teóricas o bien por experimentación, esto es debido a que 
            es fuertemente alterado por las condiciones del medio en el que se encuentre. Es importante mencionar también que
            este parámetro es medido a través de un hardware que rara vez tiene un comportamiento idéntico.
            \paragraph{}
            Los modelos de cálculo del RRSI se basan en la pérdida de señal en el espacio, como sabemos la potencia de la señal
            disminuye con el cuadrado de la distancia. Esta ecuación de Friis para la transmisión libre en el espacio espacio es 
            una fórmula teórica, existen aproximaciones obtenidas por métodos empíricos:
            \begin{equation}
                P_L(d) [dB] = P_L(d_0) [dB] + 10n ·\log_{10} \frac{ d_i }{d_0 } 
            \end{equation}
            \paragraph{}
            Siendo PL (d0)  la pérdida de propagación a 1 metro y n es una constante que depende del medio, será igual
            a 2 si se encuentra en el espacio libre sin obstáculos ni reflexiones o dispersiones de señal.
            \paragraph{}
            Para calcular este parámetro 'n' se ha de aplicar la siguiente fórmula, como podemos ver para llevar a cabo el 
            cálculo se han de tomar valores empíricos de potencias:
            \begin{equation}
                n = \frac{ P_L(d_i) - P_L(d_0) }{10n*\log_{}\frac{d_i}{d_0}}
            \end{equation}
            \paragraph{}
            Por lo tanto, y una vez obtenidos los valores de potencia a distintas distancias podemos obtener la ganacia
            de la señal recibida:
            \begin{equation}
                RRSI [dBm] = -10n*\log_{10} d+ A[dBm]
            \end{equation}
            \paragraph{}
            Disponiendo ya del valor de la constante de pérdidas, d, calculado con la ecuación (2) y los valores
            de rssi medidos desde la antena receptora a 1 metro de distancia, A, podemos obtener:
            \begin{equation}
                d= 10^\frac{-(RRSI - A) }{10n}
            \end{equation}
        \subsubsection{Cálculo de consumos energéticos}
            \begin{itemize}
                \item  Active mode: 160-260mA.  Active core, ULP coprocessor and RTC
                Wifi, Bluetooth, Radio, Peripherals
                \begin{center}
                    \begin{tabular}{||c || c ||} 
                    \hline
                    ESP32 Mode & Power consumption  \\ [0.5ex] 
                    \hline\hline
                    Wi-Fi Tx packet 13dBm~21dBm & 160~260mA  \\ 
                    \hline
                    Wi-Fi/BT Tx packet 0dBm	 & 120mA  \\
                    \hline
                    Wi-Fi/BT Rx and listening & 80~90mA  \\
                    \hline
                \end{tabular}
                \end{center}
                \item  Sleep mode: 3-20mA Active core, ULP coprocessor and RTC
                Inactive: Wifi, Bluetooth, Radio, Peripherals
                \item  Light sleep mode: the CPU is paused by powering off its clock 
                pulses, while RTC and ULP-coprocessor are kept active. This results in 
                less power consumption than in modem sleep mode which is around 0.8mA.
            
                \item Deep sleep mode, the CPU, most of the RAM and all the digital 
                peripherals are powered off. The only parts of the chip that remains 
                powered on are: RTC controller, RTC peripherals (including ULP 
                co-processor), and RTC memories (slow and fast).
                The chip consumes around 0.15 mA(if ULP co-processor is powered on) to 10µA.
            
                In Deep sleep mode, power is shut off to the entire chip except RTC module. So, any data that is not in the RTC recovery memory is lost, and the chip will thus restart with a reset. This means program execution starts from the beginning once again.
            
                \item Hibernation mode: Unlike deep sleep mode, in hibernation mode the chip disables internal 8MHz oscillator and ULP-coprocessor as well. The RTC recovery memory is also powered down, meaning there’s no way we can preserve any data during hibernation mode.
            
                Everything else is shut off except only one RTC timer on the slow clock and some RTC GPIOs are active. They are responsible for waking up the chip from the hibernation mode.
                
                This reduces power consumption even further. The chip consumes around 2.5µA only in hibernation mode.
            
                \begin{center}
                    \begin{tabular}{||c | c ||} 
                    \hline
                    ESP32 Mode & Power consumption  \\ [0.5ex] 
                    \hline\hline
                    Wi-Fi Tx packet 13dBm~21dBm & 160~260mA  \\ 
                    \hline
                    Wi-Fi/BT Tx packet 0dBm	 & 120mA  \\
                    \hline
                    Wi-Fi/BT Rx and listening & 80~90mA  \\
                    \hline
                \end{tabular}
                \end{center}
            \end{itemize}
            \begin{enumerate}
                \item  yas
            \end{enumerate}
        \subsubsection{Cálculo saturación espectro de frecuencia}
            El espectro de frecuencia es amplio y cada banda tiene su aplicación, dentro de las telecomunicaciones
            distinguimos las siguientes opciones:
            \begin{itemize}
                \item Banda ISM: Son las bandas de radio industriales, científicas y médicas, están reservadas a 
                nivel internacional para estos usos y pueden usarse sin licencias por parte de gobiernos. El uso de esta
                banda conlleva que todo equipo emisor ha de estar protegido frente a interferencias dentro de su mismo 
                rango, es decir han de ser tolerantes a errores.
    
                Desde el microondas y el mando del garaje hasta el bluetooth, NFC y wifi son ejemplos de tecnologías
                que emplean este rango que se encuentra comprendido desde   6.78 MHz hasta 244 GHz en distintos tramos,
                un ejemplo de tramos serían los siguientes:
                \begin{center}
                    \begin{tabular}{||c | c | c| c||} 
                    \hline
                    Rango de frecuencia & Bandwidth & Center Frequency & Allocation\\ [0.5ex] 
                    \hline
                    \hline
                        6.765 - 6.975 MHz &	0.03 MHz &	6.78 MHz	 & Locally \\ 
                        13.553 - 13.567 MHz &	0.014 MHz &	13.56 MHz	 & Global\\ 
                        26.957 - 27.283 MHz &	0.326 MHz &	27.18 MHz	 & Global\\ 
                        40.66 - 40.7 MHz &	0.04 MHz &	40.68 MHz	 & Global\\ 
                        433.05 - 434.79 MHz &	1.74 MHz &	433.92 MHz & EU, RU, Africa\\ 
                        902 - 928 MHz &	26 MHz&	915 MHz &	Americas\\   
                        2400 - 2500 MHz &	100 MHz &	2450 MHz & Global\\ 
                        5725 - 5875 MHz &	150 MHz &	5800 MHz & Global\\ 
                        24 - 24.250 GHz &	250 MHz &	24.125 GHz & 	Global\\ 
                        61 - 61.5 GHz &	500 MHz &	61.250 GHz & 	Locally \\ 
                        122 - 123 GHz &	1000 MHz &	122.5 GHz	 & Locally \\ 
                        244 - 246 GHz &	2000 MHz &	245 GHz & Locally \\ 
                    \hline
                    \end{tabular}
                \end{center}
                \item Banda no ISM: Para usos no licenciados: Son aquellas que no requieren permisos especiales de las 
                autoridades para su uso. A nivel mundial el 2,4GHz está dentro de las frecuencias de libre utilización 
                con ciertas limitaciones en algunos paises.

                Es por ello por lo que al ser libres podemos tener emitiendo a nuestros equipos sin preocuparnos por
                la legislación existente, la gran versatilidad para tranmitir también conlleva un mayor número de equipos 
                en este ancho de banda y mayores dificultades.
            \end{itemize}
            La frecuencia de 2.4 GHz con su estándar 802.11 b/g/n ha llegado al límite de su capacidad, se han
            detectado incluso caidas del 30\% en areas urbanas.  

            La banda ISM de 2.4GHz es, sin lugar a dudas, la que soporta un uso más intensivo, ya que es utilizada en 
            diferentes estándares de redes Wi-Fi (IEEE 802.11b/g/n) y Bluetooth, (IEEE 802.15.1), llegando en muchas 
            ocasiones a su completa saturación.
\section{Planos}
    \subsection{Plano de mecánicos}
    \subsection{Plano de eléctricos}
        \subsubsection{Esquemático del emisor beacon}
        \subsubsection{PCB circuito del emisor beacon}
        \subsubsection{Esquemático del receptor ESP32}
        \subsubsection{PCB circuito del receptor ESP32}
\section{Pliego de condiciones}
    \subsection{Prescripciones ténicas generales}
        \subsubsection{Normativa relativa a radiofrecuencia}
        \subsubsection{Normativa relativa a }
    \subsection{Prescripciones ténicas particulares}
        \subsubsection{Condiciones que ha de reunir el material}
        \subsubsection{Ejecución del proyecto}
        Placa ha de cumplir solo 1 capa
\section{Presupuestos}
    \subsection{Precios unitarios}
        \begin{center}
            \begin{tabular}{||c | c ||} 
            \hline
            ESP32 Mode & Power consumption  \\ [0.5ex] 
            \hline\hline
            Wi-Fi Tx packet 13dBm~21dBm & 160~260mA  \\ 
            \hline
            \end{tabular}
        \end{center}
    \subsection{Presupuestos Parciales}
        \subsubsection{Capitulo 1: Placa de circuito impreso emisor beacon}
        \subsubsection{Capitulo 2: Placa de circuito impreso receptor}
        \subsubsection{Capitulo 1: }
        \subsubsection{Capitulo 1: }
    \subsection{Presupuestos Total}
        con iva
    \subsection{Factores económicos y financieros}
    Estudio de viabilidad económica
        \subsubsection{Tasa de interés de retorno(TIR)}
        \subsubsection{Valor Actual Neto (VAN)}

\section{Estudio de compativilidad electromagnética}
\subsection{test}
\subsubsection{test}

\includegraphics[scale=0.3]{5min_beacon_rssi}
\includegraphics[width=15cm, height=8cm]{5min_beacon_rssi}
\paragraph{hola}
asdfasd



\section{Bibliografía}
\href{https://campus.masterd.es/campusvirtual/index.htm}{Something Linky} 





\end{document}